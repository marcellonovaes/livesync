\label{crowdsync}
A traditional video presentation involves a single User Device that is able to decode and present this single content (the Main Content) originated from a unique source. In a non-traditional scenario \cite{huang2013evolution}, the presentation environment is composed of multiple user devices (TV, smart phones, tablets, etc.) able to present multiple contents delivered for multiple sources. One such scenario is an user that access a web video and is able to access other correlated videos with different angles, audio and complementary information.

In this situation, the user accesses a mashup of digital contents that may have no explicit synchronization defined to orchestrate the  presentation. Mashups are applications generated by combining content, presentation or other applications functionalities from disparate sources. They aim to combine these sources to create useful new applications or services to users \cite{yu2008understanding}. This combination of services and contents however brings the following issue: how to synchronize these multiple contents for each user in its environment, since the contents are transmitted through different channels and from different sources that are not explicitly synchronized among them? To tackle this issue we use the crowd as part of our solution. They act as couplers, in other words, they are responsible for finding the synchronization points among related videos, allowing their synchronous presentation in a mashup video application.

The goal of video synchronization is to align a set of videos $\alpha$ in a common temporal line \cite{segundo2015remote}. For this purpose, consider $\alpha1$ and $\alpha2$ as two continuous videos. They are considered to be synchronized when, in a given time $T_{k}$ at the $kn^{th}$ time instant of $\alpha1$ and $T_{M}$ at the $mn^{th}$ time instant of $\alpha2$, they both correspond to the same instant in global time when they were captured, which is an instance of continuous space-time. If they are not synchronized, there is a time offset ($\Delta$) that added to the presentation of $\alpha1$ or $\alpha2$ will make them synchronous. The time offset between two videos V1 and V2 can be defined as $\Delta_{V1,V2} = b.V1-b.V2$, where "b.V" is the starting time of video V in reference to a timeline of the related videos, and $\Delta$ is the time offset between them in this timeline. Finding this $\Delta$ is the task attributed to the crowd. They are responsible for analysing the videos, finding the correlated ones and setting the $\Delta$ that makes them synchronous.

Three different Crowd Synchronization scenarios are presented next: Chunk Synchronization, Frame Synchronization and Live Synchronization. 