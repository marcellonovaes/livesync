Content Providers are third-parties videos streamers platforms. There are multiple Live Video Stream applications in the market, such as YouTube Live, LiveStream, TwitCast, Twitch and Ustream. One of our objective was to allow the use of different platforms as video sources, so we maximize the number of videos for an event and allow the use of already in market platforms. 

A Content Provider needs two requisites to be compatible with LiveSync: a Remote Player and Uptime Support. As each stream platform uses their own protocols, we opt to use their embeddable players into a MashUp application. These players must allow us to play, pause and stop the video stream. The second requisite, Uptime Support, is necessary to find the couplers among the videos. Uptime is the time passed since the beginning of the live stream until the video part being presented in the player at the moment of the call.

At this time, LiveSync supports two video stream platforms: the YouTube Live (https://goo.gl/DEM9eW) and WebSocket Stream (http://goo.gl/GYnGpg). YouTube live allows live stream from both desktop and mobile devices, making possible users to stream any event they wish with low effort: they only need to install the software and to have a YouTbe account. WebSocket Stream is an open source project that allows developers to implement live stream services with websockets and canvas and WebGl technologies. It works in any browser and was used for our first tests where we needed full control of the stream, something that YouTube Live doesn't support.

To add an video source form a Content Provider, it necessary only the video ID from YouTube Live or the video stream URI from the WebSocket Stream. These are added as assets to the LiveSync and can be accessed or synchronized.


%%%%%%%%%%%%%%%%% ACM %%%%%%%%%%%%%%

%\textbf{\textit{Content Providers}}

%Content Providers are third-parties videos streamers platforms. There are multiple Live Video Stream applications in the market, such as YouTube Live, LiveStream, TwitCast, Twitch and Ustream. One of our objective was to allow the use of different platforms as video sources, so we maximize the number of videos for an event and allow the use of already in market platforms. A Content Provider needs two requisites to be compatible with LiveSync: a Remote Player and Uptime Support. As each stream platform uses their own protocols, we opt to use their embeddable players into a MashUp application. These players must allow us to play, pause and stop the video stream. The second requisite, Uptime Support, is necessary to find the couplers among the videos. Uptime is the time passed since the beginning of the live stream until the video part being presented in the player at the moment of the call.