This work presents a tool that allows users to synchronize live videos from multiple sources such as YouTube or any other video streaming sources. The proposed approach to proceed the multiple camera video synchronization is based in crowdsourcing techniques, using the power of a crowd of collaborators to synchronize videos, requiring from each user the sync of only a pairs of videos. Additional sync relations are inferred from the known contributions, using transitivity properties and an appropriate structure for this inference, the Dynamic Alignment List.

%%%%%%%%%%%%%%% ACM %%%%%%%%%%%%

%User Generated Videos are contents created by heterogeneous users around an event. Each user films the event with his point of view, and according to his limitations. In this scenario, it is impossible to guarantee that all the videos will be stable, focused on a point of the event or other characteristics that turn the automatic video synchronization process possible. Focused on this scenario we propose the use of crowdsourcing techniques in video synchronization (CrowdSync). The crowd is not affected by heterogeneous videos as the automatic processes are, so it is possible to use them to process videos and find the synchronization points.  In order to make this process possible, a structure is described that can manage both crowd and video synchronization: the Dynamic Alignment List (DAL). Therefore, we carried out three experiments to verify that the crowd can perform the proposed approach: the first experiment used a crowd simulator to verify the DAL capability of managing videos and contributions, generating cohesive video presentations; the second experiment used a crowd to synchronize videos performing small tasks; the third explored the use of the crowd to synchronize Live Stream Videos, through the development of the LiveSync tool.

%%%%%%%%%%%%% NEW  %%%%%%%%%%%%%%%

%User Generated Videos (UGV) are contents created by heterogeneous users around an event where each user films the event from his point of view, and according to his limitations. This work presents a method to synchronize UGV from multiply sources as YouTube or any other video streaming source. Heterogeneous users tends to generate heterogeneous videos, making impossible to guarantee that all videos will be stable, focused on a point of the event, or other characteristics that turn the automatic video synchronization process possible. Although, humans are much less affected by videos heterogeneity than automatic methods. Based on this we proposed a crowdsourced approach, using the processment power of a crowd of collaborators to synchronize them. The proproposed crowdsourced method for synchronize live UGV streamings from multiply sources is named LiveSync. It is based on the Dynamic Alignment List (DAL), an abstract data type that can manage both crowd contributions and video synchronization. Therefore, we carried out three experiments to verify that the crowd can perform the proposed approach: the first experiment used a crowd simulator to verify the DAL capability of managing videos and contributions, generating cohesive video presentations; the second experiment used a crowd to synchronize videos performing small tasks; finally the third explored the use of the crowd to synchronize Live Stream Videos, through the development of the LiveSync tool.

