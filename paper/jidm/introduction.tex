
User generated videos (UGV) live streaming continuously grows in number end relevance boosted by platforms such as Facebook, Youtube and Vimeo. Into this scenario is relevant find out methods capable to synchronize them. 

Automatic methods are widely used to reach video synchronization. Although, they generally demand vast example databases as well controlled conditions, standardized structure and marks to work properly \cite{wang2014videosnapping}. These methods usually present good results synchronizing well-structured video productions, professional coverage for sport events and other modalities of planed videos. However, they tends to face challenges attempting to synchronize heterogenous videos in situations where have to deal with wide baselines, camera motion, dynamic backgrounds and occlusion \cite{schweiger2013fully}.

UGV are produced over the user's point of view, so different users covering a same event tends to result in heterogeneous video streams, in different angles, qualities, audio content, and other characteristics that make automatic methods unappropriated to synchronize these sources.

To achieve synchronization for UGV live streamings from different sources,  this work introduces LiveSync, a method that explore the human ability of associate  heterogeneous videos. LiveSync is based on the Human Computation paragidm \cite{VonAhn:2005:HC:1168246}, using human intelligence to execute tasks usually hard to machines but easy to ordinary people using its senses (vision and hearing). Additionally, human computation can improve performance by division of labor because it helps to define tasks that can be executed in parallel \cite{Rohwer:2010:NHC:1837885.1837897}. This characteristic is potentialized in LiveSync by adopting a crowdsourcing approach, to use efficiently the processing power of a crowd of collaborators, collecting their contributions in parallel \cite{howe2006rise} . 

Crowndourcing is an approach in which a problem is divided in tasks, and the execution of these tasks is delegated to the crowd composed by individuals engaged in the solution process. The partial result delivered by each contribution are registered, and a final outcome is generated by processing them. A very popular crowdsourcing strategy consists in to distribute tasks that can be completed easily and quickly \cite{Difallah:2015:DMC:2736277.2741685}.

The application scenario for LiveSync contains a set of UGV live streams from different sources, that must be synchronized. These synchronization issues are related to time delays between videos streamed from different sources. The occurrence of time delay between a pair of videos is manifested as misalignment, in other words, there is a time difference between scenes that should be displayed at the same time.  In this work the time delay between a pair of videos if refereed as $\Delta{time}$.

In the proposed method is created a list with all possible pairs of video streams, and each collaborator is asked to provide a synchronization point for one pair.  The delay between each pair of streams is determined by processing the contributions, and the streams are aligned based on these delays, making possible to align them in a synchronized presentation.

This paper extends the previous work "LiveSync: a tool for Real Time Video Streaming Synchronization from Independent Sources" \cite{delivesync} presented in the XXII Brazilian Symposium on Multimedia and Web Systems (WebMedia 2016) where it was honoured with the Best Paper Award at XVI Workshop of Tools and Applications (WFA).

This remain of this work is organized as follows: section 2 describes method LiveSync, section 3 details LiveSync Tool, section 4 presents an study case to evaluate the proposed method. Finally, section 5 concludes with some remarks about this work and further research.   


