
\textit{This paper extends the previous work with same title \cite{delivesync} presented in the XXII Brazilian Symposium on Multimedia and Web Systems (WebMedia 2016) where it was honoured with the Best Paper Award at XVI Workshop of Tools and Applications (WFA).}

Multiple camera synchronization is an ubiquitous theme within the multimedia area, commonly related to automatic video synchronization (AVS) methods \cite{wang2014videosnapping}. Although, these automatic methods generally demands vast example databases as well controlled conditions, characteristics and marks to work properly.  These characteristics make AVS suitable to synchronize well-structured video productions, professional coverage for sport events and other modalities of planed videos, but they tends to face challenges attempting to synchronize heterogenous videos in situations where have to deal with wide baselines, camera motion, dynamic backgrounds and occlusion \cite{schweiger2013fully}.

Into a scenario in which user generated videos (UGV) live streaming continuously grows in number end relevance boosted by platforms such as Facebook, Youtube and Vimeo,  it is relevant find out methods capable to synchronize them. UGV are generated over the user's point of view, so different users covering a same event tends to result in heterogeneous video streams, in different angles, qualities, audio content and other characteristics that make AVS unappropriated to synchronize these sources.

To achieve UGV live streaming synchronization this work introduces LiveSync, a  method that explore the human ability of associate  heterogeneous videos, to synchronize UGV live streams from different sources. LiveSync in based on the Human Computation (HC) paragidm \cite{VonAhn:2005:HC:1168246}, using human intelligence to execute tasks usually hard to machines but easy to ordinary people. 

Additionaly HC can improve performance by division of labor because it helps to define tasks that can be executed in parallel \cite{Rohwer:2010:NHC:1837885.1837897}. This characteristic is potentialized in LiveSync by adopting a crowdsourcing \cite{howe2006rise} approach to using efficiently the processing power of a crowd of collaborators, collecting their contributions in parallel. Each collaborator can contribute by simply finding a synchronization point between a pair of video streams, following a crowdsourcing strategy that consists in to offer workers tasks that can be completed easily and quickly \cite{Difallah:2015:DMC:2736277.2741685}.

This paper is organized as follows: section 2 describes LiveSync method, section 3 details LiveSync Tool, section 4 presents an practical experiment to validate LiveSync and LiveSync Tool. Finally, section 5 concludes with some remarks about this work and further research.   




