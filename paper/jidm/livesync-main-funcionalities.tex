The main functionalities provided by or tool are:
\begin{description}
	\item[Synchronized Live Video Player -]	the tool permits users to watch multiple videos synchronized. He selects from a list of sources the videos he wish to watch and them they are synchronized using information provided by other users.
	
	\item[Video Synchronization -] If a pair of videos does not have any information about their synchronization, users are invited to contribute and synchronize the videos.
	
	\item[Infer Synchronization -] In some cases that there is no direct information about the synchronization of two videos, the tool is able to infer the synchronization about them, based on the contributions of other videos. We use the transitive attribute of video synchronizations, where if we know AB and BC synchronization info, we can infer AC. To infer this value we travel trough the DAL, finding the unknown relations (for example, CE), and try to find a path of known relations where we can infer CE. Taking the DAL in Figure~\ref{dal}, we can infer CE if we know: AC and AE. This a two steps rout, but try all possible routes when inferring, in a way that we fill as much relations as possible.
	
	\item[Video Aggregation -]	Although the focus of the LiveSync tool being on synchronization, we allow users to add new stream sources to the application. He only needs to set the video source, and the video will be added to the DAL and list of videos. However, we don't do any filtering about the added video, this means that the user can add any video to the application, even ones that contains none relation with the other videos. In future versions we plan to add options where other users can mark the video as not related, and then remove them.
	
	\item[Multiple Platforms Support -] One keypoint of our tool is the use of other platforms as video sources. The videos that we play to users and that we synchronize, are not provided by us, but by other live video stream platforms. To be compatible with our tool, two requisites are required:
	\begin{enumerate}
		\item Remote Player: we need that the platform allows embeddable player on third pages, allowing us to control the player with its basic functionalities such as: play, pause and stop;
		\item Uptime Support: a second and fundamental requisite is an API that allows us to retrieve the video Uptime. Video uptime is the time since the beginning of the video that is presented on the video player. This is fundamental to create and replicate the couplers generated in synchronization process.
	\end{enumerate}
	
	\item[Serverless Architecture -] Serverless architectures refer to applications that significantly depend on third-party services and putting much of the application behavior and logic on the front end. Such architectures remove the need for the traditional server system sitting behind an application \cite{RobertServerless}. More of this characteristic will be addressed in the next topic: Architecture.
	
	\item[Multiplatform -] LiveSync is a Web Based application designed and developed in compatibility with HTML5 standard to its front-end (MashUp Player) component. It allows our application to run on multiple browsers, operational systems and devices.
	
	\item[Active X Passive Contributions -] Two branches of the LiveSync are currently on our repositories. They differ only in one aspect: who what videos are to be synchronized, the crowd or the application? The active version allows user to navigate freely through the videos, synchronizing them when they wish to. The focus of this branch is to allow users to contribute if they want to. On the other hand, the Passive branch gives the crowd exactly what video they will synchronize. The focus here is to rapidly synchronize all videos, so the focus isn't to make users watch the videos, but force them to synchronize all the base for other porpoises. The active branch is the focus here, but can easily converted to the passive.
	
\end{description}