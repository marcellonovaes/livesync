The Coupler is responsible for storage, distribution and calculation of synchronization points among videos streams from the Content Providers.

A coupler is composed of a DAL instance and Log files. This goes in direction of the Serverless Architecture. We wanted an architecture that needed low resources (another justification for using third party stream services) and easy deployment. All that is necessary do execute the coupler is a NODE.JS (https://nodejs.org/en/) server instance. This is possible because the couple is fully developed in  and compatible with actual HTML5 standards. To deploy the Coupler, we use a Backend as a Service or "BaaS" platform, more specifically we use the Heroku one, that permits free use of NODE.JS instances.

It stores synchronization information only during the duration of the event, so its stance is finished with the end of the videos and all data is lost. In the current scope, the sync info is only necessary during the event, after it, there is no need to store the information. For reasons of testing and using the filmed videos from YouTube we create log files that contains all contributions made by the crowd. If it is important to maintain all contributions and data for post analyses and further use, unstable version of the LiveSync is being configured to use a fully transactional database. we use a fully transactional database because we want to maintain track of all contributions made by the crowd, something important in crowdsourcing and that is supported also by the DAL.

Other aspect of the Coupler is that it is responsible for the distribution of synchronization couplers. When an user chooses two videos, a message is sent from the MashUp to the coupler, containing the required relation. The coupler then answer with the required information. If the relations is unknown, it answer soliciting the user to synchronize and contribute with those two videos.

The last function of the Coupler, is to calculate the synchronization points among Videos. Each relation ($\Delta_{A,B}$) may contain several contributions, then it is necessary to calculate a value to that relation based on the contributions. In the current version we calculate a Geometric Mean of the contributions to find an ideal value. This however may not be the best value, because the more accurate the sync is, the better results, so older contributions must have a smaller weight, something that does not happen now. The other calculation made by the coupler, is to infer unknown relations based on the afore mentioned transitive relation.

The communication Coupler - MashUp is made through Websocket [REF] communication. The MashUp creates a WebSocket channel with the Coupler, and requests the sync information or sends contributions from the crowd. A simple protocol is used in Json messages: {act:value, data:object}. The act field contains the action to be made and the data contains an object to complement the action. As example we have an act to send a new contribution ("contribution") that is complemented with a new relation that contains the assets involved, the value and an id to that contribution.