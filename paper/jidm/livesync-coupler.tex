The Coupler Service (CS) is responsible for storage, distribution and calculation of synchronization points between video streams from the Content Providers. Coupler Service provides a DAL instance and Log files. 


%This goes in direction of the Serverless Architecture. 



%We wanted an architecture that needed low resources (another justification for using third party stream services) and easy deployment. All that is necessary do execute the CS is a NODE.JS (https://nodejs.org/en/) server instance. This is possible because the CS is fully developed in JavaScript and compatible with the HTML5 standards. To deploy the CS, we use a Backend as a Service or ”BaaS” platform, more specifically we use the Heroku (www.heroku.com) one, that permits free use of NODE.JS instances.


It stores synchronization information only during the duration of the event, so its stance is finished with the end of the videos and all data is lost. In the current scope, the sync info is only necessary during the event, after it, there is no need to store the information. For reasons of testing and using the filmed videos from YouTube was generated log files that contains all contributions made by the crowd. If it is important to maintain all contributions and data for post analyses and further use, unstable version of the LiveSync is being configured to use a fully transactional database.  A fully transactional database is used to track all contributions made by the crowd, an important aspect for crowdsourcing support and that is also supported by the DAL.

Other aspect of the CS is that it is responsible for the distribution of relational couplers, containing synchronization information between pairs of video streams. When an user selects a pair of streams, a message is sent from the mash-up to CS, containing the required relation. CS then answers it with required information. If the relation is unknown, it responds with a prompt for the user to synchronize them.

The last functionality on Coupler Service, is to calculate the synchronization points between videos streams. Each relation ($\Delta_{A,B}$) may contain several contributions, then it is necessary to calculate a value to that relation based on the contributions. In the current version, this delta is calculated by the geometric mean of the contributions to find an satisfying value. CS also uses the Inference functionality by DAL to find additional relations over these calculated values.

The communication  between CS and the mash-up application is made through Websocket communication. The mash-up application creates a WebSocket channel to CS, and requests the sync information or sends contributions from the crowd. A simple protocol is used for JSON messages: {act:value, data:object}. The \textit{act} field contains the action to be made and the \textit{data} field contains an object to complement the action. As example, the action used to send a new contribution, transmit as value for its \textit{data} field a relational coupler representation that contains the videos involved, as so the contribution value for the $\Delta$ between them.

