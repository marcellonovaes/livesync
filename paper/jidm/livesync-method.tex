In a live presentation, it is assumed that content must be consumed right after its generation. It is of extreme importance then, that the synchronization method can be performed in playback time, to allow the integration of live content.

The way synchronization is performed in live video poses a single requirement: in a live situation, there is no need to analyse an entire video to find synchronization points, only an estimated stream delay must be considered. This occurs because the event is happening in real time (live), sources are also live and, therefore, the lack of synchronization during playback is caused by video streaming delays. In this specific case, synchronization through the crowd may be achieved by using tools that allow a user to manipulate time from videos: he can, for instance, keep pausing videos alternately , until he feels they are synchronized. This is a consequence of having a reduced search space, when limited to stream delays between videos.

However, not all viewers are required to be part of the crowd to achieve synchronization. If a synchronization made by a single member of the crowd is accepted as accurate, it can be transferred to the remaining viewers, this way each one will have his content locally synchronized.

One way of live synchronization can work is as follows: a person selects and synchronizes two videos with the help of a manipulation tool. This becomes a candidate synchronization point. Several people can do the same, and the results can be based on multiple synchronizations. Having these synchronization points defined, synchronization information can be sent to other viewers interested in watching those videos. As simple example, take two independent sources that are transmitting an event. A mash-up system allows the user to watch both videos at the same time is his device. However the videos are asynchronous and the user notes that. He then access the option to synchronize the videos. After he achieve a synchronous result, implicitly his contribution is sent to a server that will feed other users that choose to watch the same videos with the synchronization specification. If the user thinks the content is not synchronised yet, he can synchronize it himself and send another contribution.