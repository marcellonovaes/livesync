In Quantification of YouTube QoE via Crowdsourcing \cite{hossfeld2011quantification} the crowd is used when evaluating YouTube's quality of services (QoS). Each crowd individual watches videos from his home and rates his experience. An important point of the paper is the crowd filtering: in multiple tests executions, nearly 80\% of them were removed from results as they were considered to fail the evaluation. For the elimination, these techniques were applied: Golden Standard Data; Consistency Tests; Content Quests; Mixed Answers; and Application Usage Monitoring.
Video quality evaluation in the cloud \cite{keimel2012video} also presents a quality evaluation, but this time an evaluation from the actual video.

A subjective evaluation using crowdsourcing of Adaptive Media Playout utilizing audio-visual content features \cite{rainer2014subjective} presents another paper that uses the crowd to evaluate video quality. In this case, issues concerning adaptive video are evaluated. It is asked to participants to evaluate video quality, without knowing if video quality was modified or it remains the same.