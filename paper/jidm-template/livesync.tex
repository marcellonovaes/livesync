Live synchronization includes the scenario where a viewer has access to an event that is live streamed by more than one Content Provider. These content providers are independent, so their videos do not have initial resources that allow their automatic synchronization to viewers, requiring a video analysis to generate synchronization points (Couplers). This synchronization fits as a problem that can be solved by using the power of the crowd. This occurs because videos are generated independently, without synchronization points and without previous description of what is about to be shown in screen, neither how can it be correlated to other videos. This way, human perception is used in real time to generate unknown synchronization points. 

As example for this scenario, we can take a public manifestation. In the event, multiple people can take their cell phones and start streaming the event. In their house, other people can watch the videos. However, the multiple videos from different sources will be asynchronous. We need a way to synchronize these UGV. We use the crowd to achieve it. To this objective we group all videos in a MashUp application that connects to a Coupler Server that contains all synchronization data. Both synchronizing and playing the videos are made using this mashup, that can receive videos from multiple sources.






%%%%%%%%%%%%%%%%%%% ACM %%%%%%%%%%%%%%

Live synchronization includes the scenario where a viewer has access to an event that is live streamed by more than one Content Provider. These content providers are independent, so their videos do not have initial resources that allow their automatic synchronization to viewers, requiring a video analysis to generate synchronization points (Couplers).

As example for this scenario, we can take a public manifestation. In the event, multiple people can take their cell phones and start streaming the event. In their house, other people can watch the videos. However, the multiple videos from different sources will be asynchronous. We need a way to synchronize these UGVs. We use the crowd to achieve it. To this objective we group all videos in a mashup application that connects to a Coupler Server that contains all synchronization data. Both synchronizing and playing the videos are made using this mashup, that can receive videos from multiple sources.

In a live presentation, it is assumed that content must be consumed right after its generation. It is of extreme importance that the synchronization method can be performed in playback time, to allow the integration of live content. However, not all viewers are required to be part of the crowd to achieve synchronization. If a synchronization made by a single member of the crowd is accepted as accurate, it can be transferred to the remaining viewers, this way each one will have his content locally synchronized.

One way of live synchronization can work is as follows: a person selects and synchronizes two videos with the help of a manipulation tool. This becomes a candidate synchronization point. Several people can do the same, and the results can be based on multiple synchronizations. Having these synchronization points defined, synchronization information can be sent to other viewers interested in watching those videos. As simple example, take two independent sources that are transmitting an event. A mashup system allows the user to watch both videos at the same time is his device. However the videos are asynchronous and the user notices that. He then access the option to synchronize the videos. After he achieve a synchronous result, implicitly his contribution is sent to a server that will feed other users that choose to watch the same videos with the synchronization specification. If the user thinks the content is not synchronised yet, he can synchronize it himself and send another contribution. This tool was implemented and is presented on section~\ref{livesync}.





