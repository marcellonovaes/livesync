The Contribution Management is a requisite in order to adequate the DAL for a crowdsourced scenario. It is necessary to receive, distribute and process the crowd contributions in order to find the correct Time Offsets. The aspects involved in the Contribution Management are:

	\textbf{Convergence:} one of the principles of crowdsourcing is collecting the contribution of multiple members and based on these contributions finding the solution to a problem. In our case, the crowd watches the videos fragments and find synchronization points. The convergence is responsible for getting all theses contribution, and merging them in actual results.
	
	The Convergence Level detects the tendency of the Crowd about the Relation of a pair of videos. This tendency is an indicative of the agreement on a Delta between the videos. According to the scenario, a Convergence Threshold that is used to determine if a Relation is converged, is defined. Each Relation have an attribute that stores its current convergence level. If this level reaches a Convergence Threshold the method returns True, otherwise it returns False.

	Initially all Relations are created with Convergence Level 0, and consecutive similar contributions increase this value. If a tendency change in the contributions is detected, the Convergence Level is reset.
	
	\textbf{Video Selection:} the convergence deals with the problem of receiving and processing the crowd contributions. However, we must correctly choose the videos to be evaluated by the crowd that results in a contribution. The selection of these videos is part of the contribution management. 
	
	This method intent is to select which pair of videos should have priority to receive contributions. It consists of a sequence of two other methods: (i) Choose the next Video; (ii) Choose the next Relation. When this method returns NULL, it means that the DAL has been converged.
	
In order to increase possible inferences over the contributions, the method try to spread contributions over the timeline, proceeding a random selection among the Videos that were not converged nor marked as a Impossible relation. Once a Video is chosen the method select from its Relations array the Relation closer to converge.