Aiming to verify the CrowdSync method and the data structure generated to implement the method, two different experiments were executed:

\begin{description}
	\item[LiveSync Tool:] the LiveSync Tool implements all requisites for live synchronizing live UGV streams. Its objective is to allow the creation of mash applications of related video streams with client based solutions.
	\item[Crowd Simulated DAL:] the second experiment uses a simulated crowd to verify the DAL. The objective here is to verify if the DAL correctly stores, infers, converges and selects videos and relations. We simulate the crowd so we can make a controlled analysis in a set of videos with different crowd profiles.
	\item[UGV Dataset Synchronization:] in the third experiment, real users (crowd workers) synchronize a user generated video dataset using a developed platform with all characteristics described in this paper. Since the simulated experiment allows only to verify the correct functioning of the algorithms and structure, the focus here is to verify if humans using their perception can find the synchronization points in all related videos.
\end{description}




