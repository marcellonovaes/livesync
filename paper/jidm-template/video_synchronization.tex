Besides studying the relation between crowds and videos segments, it is important to consider other video synchronization techniques, and that is what we now describe now in this section.

The audio analyses can be used to synchronize the video segments. Su \textit{et al.} \cite{su2012making} presents video synchronization using this approach. Fingerprints are generated for each audio, and a comparison between videos is performed. Bano \textit{et al.} \cite{bano2015discovery} also use audio as synchronization track using the chroma analysis from audio to group and synchronize audio from a same event.

There are also the approaches where video analyses is used instead of audio. Wang \textit{et al.} \cite{wang2014videosnapping}, synchronizes videos in space and time, allowing the navigation between videos by resemblance and time. The synchronization works for multiple videos and different cameras. Other important works is described by Schweiger \textit{et al.} \cite{schweiger2013fully} that a research for related papers in the area. It presents important results in automatic video synchronization, such as a technique that analyses differences between frames to find synchronization points. Furthermore, it presents the main challenges for automatic synchronization algorithms: wide baselines, camera motion, camera shaking, dynamic backgrounds and occlusions.

The use of human perception however, is not impaired by these. The human processing can overcame all these challenges and with or without the sound information. This is the principle that guided us to the proposed CrowdSync. Here we don't claim that the CrowdSync overcomes the automatic techniques, but with the crowd we find less limitations on what videos we can synchronize.