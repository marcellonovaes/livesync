
Crowdsourcing is often a highly structured process from an organization, drawing on the creativity and intelligence of an online community in an open, but controlled, way \cite{brabham2014crowdsourcing}. Crowdsourcing systems can be used in a variety of situations such as: Knowledge Discovery and Management; Distributed Human Intelligence Tasking Organization; Broadcast Search; Peer-Vetted Creative Production \cite{brabham2014crowdsourcing}. The CrowdSync problem can be classified as a "Distributed Human Intelligent Tasking" problem, where an organization sends task related to information analysis. This kind of system deals with an information management problem: how the crowd will use information contained in videos to generate synchronization points. A second characteristic for our problem is that we consider that the information is already available and it is unnecessary to locate it. We don't approach the problem of collecting the videos, here we consider only synchronizing them.

Synchonizing videos isn't the only way to use the crowd with videos. Several works use the crowd with other objectives on videos. Two main problems may be highlighted here: annotation and quality evaluation, but diverse works can be addressed.



